\newpage \chapter{Introdução ao curso}\setcounter{SteP}{1}

    Seja bem-vindo ao {\bf Curso de Drupal}. Este curso esta sendo
fomentado pelo {\it Xxxxx}
da {\it Zzzzz} e ministrado por {\it Reinaldo Gil Lima
de Carvalho}. Realiza(ou)-se de {\it xx/yy/2017 a xx/yy/2017}.

    Os procedimentos descritos neste material foram validados sob a
distribuição GNU/Linux Debian Stretch~\cite{Debian}, todavia a base teórica ministrada é o
conhecimento fundamental para a aplicabilidade desses procedimentos sob
qualquer sistema operacional. Busca-se que o participante esteja apto a
utilizar o conhecimento adquirido no ambiente que lhe for mais adequado.

    Neste capítulo, serão abordados os seguintes temas: licenciamento deste
material, origem do {\it software} utilizado (executável ou do código fonte)
e configurações essenciais da distribuição Debian GNU/Linux.

\section{Licença do material}\setcounter{SteP}{1}

    Todas as marcas registradas são de propriedade de seus respectivos
detentores, sendo apenas citadas neste material educacional.

    O ministrante nem a fomentadora responsabilizam-se por danos causados
devido a utilização das informações contidas neste material. Não há
garantias de que este material está livre de erros, assim como, todos os
sistemas em produção devem possuir {\it backup} antes de sua manipulação.

    Este material esta licenciado sobre a {\bf GNU Free Documentation
License - GFDL} ou {\bf Licença de Documentação Livre GNU} conforme descrito a
seguir:

\begin{BoxVerbatim}
    Copyright (c) 2017 Reinaldo Gil Lima de Carvalho - reinaldoc@gmail.com

    É garantida a permissão para copiar, distribuir e/ou modificar este documento
sob os termos da Licença de Documentação Livre GNU (GNU Free Documentation
License) Versão 1.3, publicada pela Free Software Foundation; com todas Seções
Secundárias Invariantes.
\end{BoxVerbatim}

    A {\bf Licença de Documentação Livre GNU} permite que todo conteúdo
esteja livre para cópia e distribuição, bem como que a propriedade autoral
seja protegida. O objetivo é garantir que o conhecimento seja livre, assim
como garante o reconhecimento ao autor.

     O autor recomenda ainda que este material seja sempre distribuído
"como está", no formato original. Contribuições e sugestões de melhorias
sobre este material podem ser enviadas ao autor e serão consideradas para
aprimoramento do material.

\section{Compilação de {\it software} X {\it software} da distribuição/sistema operacional}\setcounter{SteP}{1}

    O acesso ao código fonte do {\it software} e sua compilação, é uma das
liberdades propiciadas pelo {\it software livre}. Entretanto, o {\it software}
também pode ser obtido em forma executável (compilada), e de forma integrada
ao sistema operacional (empacotado), já estando pronto para utilização.
Cada uma destas opções possui vantagens e desvantagens que serão enumeradas
a seguir:

    Características do {\it software} obtido na forma de executável (previamente compilado):

\begin{itemize}
\item{\bf } {\bf V:} Instalação rápida que requer menos espaço em disco; evita a
compilação do {\it software}, assim como, a instalação de {\it software} de compilação
(make, gcc, etc) e cabeçalhos de bibliotecas (libc6-dev, etc)

\item{\bf } {\bf V:} Instalação automatizada de {\it software} e de bibliotecas
necessárias (dependências) para o funcionamento do {\it software} principal.

\item{\bf } {\bf V:} Versão testada pelo distribuidor do {\it software} (em geral
o distribuidor do sistema operacional), e possivelmente livre de erros.

\item{\bf } {\bf V:} Possibilita atualizações e correções de falhas de segurança
de forma automática, e fornecida pelo distribuidor do sistema operacional.

\item{\bf } {\bf V:} Facilita suporte externo devido ao método de instalação
padronizado e utilização de versões invariantes do {\it software}.

\item{\bf } {\bf V/D:} Pode não ser a versão mais nova do {\it software}, e não
possuir funcionalidades mais recentes. Todavia, a utilização de versões 
maduras, tende a fornecer maior estabilidade.

\end{itemize}


    Características do {\it software} obtido a partir do código fonte:

\begin{itemize}
\item{\bf } {\bf D:} Instalação mais complexa e demorada, demanda instalação
manual de bibliotecas externas.

\item{\bf } {\bf D:} Atualizações e correções são manuais, exigindo atenção
diária às atualizações necessárias para correções de falhas de segurança.

\item{\bf } {\bf D:} Dificulta suporte externo pois não é um método de
instalação padronizado.

\item{\bf } {\bf V/D:} Permite utilizar a última versão do {\it software},
com os novos recursos, mas trata-se de código menos testado podendo possuir
falhas não detectadas.

\item{\bf } {\bf V:} Pode permitir um ganho de performace com a compilação com
otimizações do processador, e também com o desligamento de recursos não
utilizados do {\it software}.

\end{itemize}

    Após o levantamento destas características, é notável que em ambientes
corporativos a utilização de {\it software} fornecido por um distribuidor é
essencial para continuidade da disponibilidade dos sistemas.

    Diminui-se o esforço empregado para manter o parque tecnológico atualizado
e livre de falhas. Dessa forma, o treinamento utilizará os pacotes fornecidos
pelo distribuidor do sistema operacional escolhido.

\section{Debian GNU/Linux}\setcounter{SteP}{1}

    Os sistemas operacionais desenvolvidos a partir de tecnologias livres
fornecem ferramentas que realizam instalações automatizadas de {\bf pacotes
{\it software}}. O Debian utiliza o {\bf apt-get} para este fim e fornece
repositórios web que contém pacotes de {\it software} disponíveis para
instalação. Os pacotes são contém arquivos compactados com rotinas de
pré/pós instalação e remoção, além de informações sobre dependências.

    O comportamento padrão do utilitário {\it apt-get}, ao instalar um
{\it software}, é realizar a instalação das {\bf dependências}, e também
daqueles pacotes especificados como {\bf recomendados}. Entretanto, isto
ocasiona a instalação de pacotes não requeridos e demanda a utilização de
espaço em disco adicional.

    A instalação automática de {\it software} recomendado pode ser
desabilitada por meio da adição da configuração abaixo ao arquivo
{\bf /etc/apt/apt.conf}:

\begin{BoxVerbatim}
APT::Install-Recommends "0";
\end{BoxVerbatim}

    A configuração da fonte dos pacotes a serem instalados é realizada
no arquivo {\bf /etc/apt/sources.list}:

\begin{BoxVerbatim}
deb http://ftp.br.debian.org/debian stretch main contrib non-free
deb http://security.debian.org/ stretch/updates main contrib non-free
\end{BoxVerbatim}

    Caso a conectividade seja fornecida por um proxy via http, a seguinte
configuração deve ser adicionada ao arquivo {\bf /etc/apt/apt.conf}, com a
devida adequação ao endereço IP do servidor proxy:

\begin{BoxVerbatim}
Acquire::http::Proxy "http://172.16.0.1:3128/";
\end{BoxVerbatim}

    Após a definição das fontes, é necessário o {\it download} da lista de
pacotes de {\it software} disponíveis, que é formada por informações de
versão e descrição. Esse {\it download}, bem como a atualização do sistema,
podem ser realizados pelos comandos:

\begin{BoxVerbatim}
    # apt-get update
    # apt-get upgrade
\end{BoxVerbatim}

    A lista de {\it software} disponíveis pode ser consultada, como indicado
no exemplo abaixo:

\begin{itemize}
\item{\bf }Pesquisar pelo nome do {\it software}:
\end{itemize}

\begin{BoxVerbatim}
    # apt-cache search ^apache2
\end{BoxVerbatim}

    Maiores informações sobre um determinado {\it software} podem ser obtidas
como indicado a seguir:

\begin{BoxVerbatim}
    # apt-cache show apache2
\end{BoxVerbatim}

